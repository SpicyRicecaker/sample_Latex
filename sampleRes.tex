%Defines a bunch of things, including paper size, font size, and type of writing
\documentclass[letterpaper, 12pt]{article}
%Sets the margin to 1 in for that cinematic view
\usepackage[margin=1in]{geometry}
%Symbols required to make equations
\usepackage{amsmath}
\usepackage{amssymb}
%Adds color to text
\usepackage{xcolor}
%Allows you to add images
\usepackage{graphicx}
%Centers captions for images
\usepackage[center]{caption}
%Allows suport for hyperlinks
\usepackage{hyperref}
%Headers and footers
\usepackage{fancyhdr}

%First defining the pagestyle to the package that we just added (fancyhdr)
\pagestyle{fancy}
%Then reset the default fancy pagestyles
\fancyhf{}
%Defining our header
\rhead{
    Bob Joe
    163 Minecraft PvP
}
%Defining out footer
\rfoot{
    %\thepage displays the current page number, so we add "Page" before that
    Page \thepage
}

%This package makes it so that all paragraphs are indented first
\usepackage{indentfirst}
%Sets the paragraph indent length to 2 font widths
\setlength{\parindent}{2em}

%Begins the body of the document
\begin{document}
%Title, author, and date are all part of a "title"
\title{Solving some random equation}
\author{by Bob Joe}
\date{19 April 2020}
%You need maketitle, otherwise it will throw an error
\maketitle

%Section header
\section{Intro}
Hello, Bobby, that is a very good problem. John approached the problem by using cross-sections perpendicular to the $x-axis$, making semicircles as stated in your word problem. However, I think that another viable way to approach the problem is to use cross-sections perpendicular to the $y-axis$, which would then make full circles circles.

%Section header
\section{Solution}
%Creates a place where you can write equations, also using & to align equations on different Lines.
%\\ is newline while within an equation
%\pm is plus or minus
\begin{align}
    \intertext{
        I found that the equation of a circle with a radius of $10$ has an equation of
    }
    x^2+y^2 & =10^2
    \intertext{Next, I wanted to make the equation of the circle in terms of $y$}
    x^2+y^2 & =10^2               \\
    x^2     & =10^2-y^2           \\
    x       & =\pm\sqrt{10^2-y^2}
    \intertext{For now, since the right side of a circle yields the same value as the left, we can just simplify the $\pm$ to $2$}
    x       & =2\sqrt{10^2-y^2}
\end{align}
Like John, I then made a desmos graph to visualize the problem. \par
%To insert an image, use \begin{figure}. The [h] stands for "here"
\begin{figure}[h]
    %Centers the image
    \begin{center}
        %Includegraphics actually adds the image in
        \includegraphics[scale=.3]{breadContainer.png}
        %href and textcolor fall under the template of \some_Command{attribute}{actual text}
        \caption{\textit{Bread Container.} Desmos link \href{https://www.desmos.com/calculator/lit7eutdys}{\textcolor{blue}{here}}, also see embed below}
    \end{center}
\end{figure}
%\boxed{} creates a box around the specified objects.
%The symbol for an integral is \int, _ is undercase, and ^ is upper case, so integrals are in the format \int_{a}^{b}
\begin{align}
    \intertext{We can define the $A(x)$ function of a circle in terms of $r$}
    A(r)                          & =\pi r^2
    \intertext{Keep in mind that the radius is half the horizontal}
    r                             & = \frac{1}{2}x
    \intertext{Substitute $x$ with our equation of the circle, which simplifies things}
                                  & = \frac{1}{2}(2\sqrt{10^2-y^2})                                                                                                     \\
                                  & = \sqrt{10^2-y^2}
    \intertext{Now we can substitute the $r$ in $A(r)$ to get $A(y)$}
    A(y)                          & =\pi (\sqrt{10^2-y^2}
    )^2                                                                                                                                                                 \\
                                  & =\boxed{\pi(10^2-y^2)}
    \intertext{Since the container is a semicircle, the bottom is $0$ and the top is $r$}
    r                             & = 10                                                                                                                                \\
    a                             & = \boxed{0}                                                                                                                         \\
    b                             & = r                                                                                                                                 \\
                                  & = \boxed{10}
    \intertext{Now, we can just integrate the area function from $0$ to $10$ to get our volume!}
    \int_{a}^{b}A\left(y\right)dy & =\int_{0}^{10}\pi(10^{2}-y^{2})dy                                                                                                   \\
                                  & =\pi\int_{0}^{10}(10^{2}-y^{2})dy                                                                                                   \\
                                  & =\pi\left(10^{2}y-\frac{y^{3}}{3}\right)\Big|_{0}^{10}                                                                              \\
                                  & =\pi\left(10^{2}\left(10\right)-\frac{\left(10\right)^{3}}{3}-\left(10^{2}\left(0\right)-\frac{\left(0\right)^{3}}{3}\right)\right) \\
                                  & =\pi\left(10^{2}\left(10\right)-\frac{\left(10\right)^{3}}{3}\right)                                                                \\
                                  & \approx 666.67\pi                                                                                                                   \\
                                  & \approx \boxed{2094.40cm^3}                                                                                                         \\
\end{align}
\section{Conclusion}
Compared to Joe's solution, I found it very interesting how in step 16 our integrals looked nearly identical to each other, disregarding the differences between $y$ and $x$. Looking at all the steps, I've made the conclusion that solving $dy$ and $dx$ is about the same difficulty and efficiency. However, $dx$ could definitely take more steps if you do not recognize that $\int_{-10}^{10}$ can be rewritten as $2\int_{0}^{10}$. I also find that doing circular cross-sections are slightly more intuitive. What do you think?
\end{document}